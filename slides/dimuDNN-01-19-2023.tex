% dinupa3@gmail.com
% 01-18-2023
%

\documentclass[12pt, xcolor={dvipsnames}, aspectratio = 169, sans,mathserif]{beamer}


\usepackage{fontspec}
\usepackage{fontawesome5}
\usepackage{mathrsfs}
\usepackage{amsmath}
\usepackage{graphicx}
\usepackage{hyperref}
\usepackage[absolute,overlay]{textpos}
\usepackage[font=tiny]{caption}


\mode<presentation>
{
\usefonttheme{serif}
\setmainfont{JetBrains Mono}
\definecolor{nmsured}{RGB}{137,18,22} % custom colors
\setbeamercolor{title}{bg=White,fg=nmsured}
\setbeamercolor{frametitle}{bg=White,fg=nmsured}
\setbeamercolor{section number projected}{bg=nmsured,fg=White}
\setbeamercolor{subsection number projected}{bg=nmsured,fg=White}
\setbeamertemplate{items}{\color{nmsured}{\faAngleDoubleRight}}
\setbeamertemplate{section in toc}[square]
\setbeamertemplate{subsection in toc}[square]
\setbeamertemplate{footline}[frame number]
\setbeamertemplate{caption}[numbered]
\setbeamerfont{footnote}{size=\tiny}
\setbeamercovered{invisible}
\usefonttheme{professionalfonts}
%\setbeamertemplate{background}[grid][color=nmsured!15] % set background
\setbeamertemplate{navigation symbols}{} % remove navigation buttons
}

\title{Deep Neural Network to Extract the Dimuon Properties}


\newcommand{\leftpic}[2]
{
\begin{textblock}{7.0}(0.5, 1.0)
\begin{figure}
    \centering
    \includegraphics[width=7.0cm]{../imgs/#1.png}
    \caption{#2}
\end{figure}
\end{textblock}
}

\newcommand{\rightpic}[2]
{
\begin{textblock}{7.0}(8.0, 1.0)
\begin{figure}
    \centering
    \includegraphics[width=7.0cm]{../imgs/#1.png}
    \caption{#2}
\end{figure}
\end{textblock}
}


\begin{document}

\begin{frame}
	\maketitle
\end{frame}

\begin{frame}[fragile]{Introduction}

\begin{textblock}{15.0}(0.5, 2.0)
\begin{itemize}

	\item We use the reconstructed single track information of the Drell-Yan events to train the neural network.

	\item Input tensor features: charge, position at station 1 drift chambers, momentum at station 1, position at station 3, momentum at station 3.

	\item Target tensor features: dimuon vertex position, dimuon vertex momentum and dimuon mass.

	\item Data set was split to train: validate: test = 60: 20: 20.

	\item Our main goal is to train the neural network to extract the dimuon vertex information.

\end{itemize}
\end{textblock}

\end{frame}

\begin{frame}[fragile]{Neural Network Architecture}

\begin{textblock}{15.0}(0.5, 2.0)
\begin{itemize}

	\item Feed-forward deep neural network witch contains 2 blocks. Classification block will try to identify the origin of the tracks and regression block will try to extract the dimuon features.

	\item Classification block;

	\begin{itemize}

		\item Contain 2 hidden linear layers.

		\item In the forward pass all the layers are activated by the \verb|ReLu| activation function.

		\item In the back propagation loss is calculated by \verb|CrossEntropyLoss|.

	\end{itemize}

	\item Regression block;

	\begin{itemize}

		\item Contains 2 hidden linear layers.

		\item In the forward pass all the layers are activated by the \verb|ReLu| activation function.

		\item In the back propagation loss is calculated by \verb|MSELoss|.

	\end{itemize}

\end{itemize}
\end{textblock}

\end{frame}

\begin{frame}[fragile]

\leftpic{dimuNet}{Neural network architecture.}

\begin{textblock}{15.0}(0.5, 9.0)

\begin{itemize}

	\item Total loss is calculated;

\begin{verbatim}
total loss = loss clas. + alpha * loss reg.
\end{verbatim}

\verb|alpha| is a non trainable hyper parameter.

	\item We use the batch training to train the neural network with batch side = 64 for 200 epochs.

\end{itemize}

\end{textblock}

\begin{textblock}{7.0}(8.0, 1.0)

\begin{itemize}

	\item Total trainable parameters = 10902 and training data size = 1519596. Rule of thumb training data size = 10* total trainable parameters.

\end{itemize}

\end{textblock}

\end{frame}

\begin{frame}{Loss Curves}

\leftpic{cls-loss}{Classification loss for each epoch.}

\rightpic{reg-loss}{Regression loss for each epoch.}

\end{frame}

\begin{frame}{Classification}

\leftpic{cls-hot-id}{Prediction of the classification.}

\begin{textblock}{7.0}(8.0, 2.0)

\begin{itemize}

	\item Tracks are coming from colimeter(id = 0), target (id = 2) and beam dump (id = 4) are predicted well. But tracks are coming from air (id = 1 and 3) region has a bad prediction.

\end{itemize}

\end{textblock}

\end{frame}

\begin{frame}{Predictions}

\leftpic{z[cm]}{z vertex position.}

\rightpic{x[cm]}{x vertex position.}

\end{frame}

\begin{frame}

\leftpic{y[cm]}{y vertex position.}

\rightpic{px[GeV]}{px at the vertex.}

\end{frame}

\begin{frame}

\leftpic{py[GeV]}{py vertex position.}

\rightpic{pz[GeV]}{pz at the vertex.}

\end{frame}

\begin{frame}

\leftpic{mass[GeV]}{dimuon mass.}

\begin{textblock}{7.0}(8.0, 2.0)

\begin{itemize}

	\item Since tracks are unique we can use the constitutional neural network for the classification. But even with the input channel = 1, CNN fails the classification.

	\item Batch normalization and Dropout layers also reduce the accuracy of the results (some how ?)

\end{itemize}

\end{textblock}

\end{frame}

\end{document}
