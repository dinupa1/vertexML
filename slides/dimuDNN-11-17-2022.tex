% author Dinupa Nawarathne
% email dinupa3@gmail.com
% date 10-11-2022

\documentclass[10pt, xcolor={dvipsnames}, aspectratio = 169, sans,mathserif]{beamer}

%
% packages
%
\usepackage{fontspec}
\usepackage{fontawesome5}
\usepackage{mathrsfs}
\usepackage{amsmath}
\usepackage{graphicx}
\usepackage{hyperref}
\usepackage[absolute,overlay]{textpos}
\usepackage[font=tiny]{caption}


%
% customization
%
\mode<presentation>
{
\usefonttheme{serif}
\setmainfont{Comic Sans MS}
% custom colors
\definecolor{nmsured}{RGB}{137,18,22}
% custom fonts
\setbeamercolor{title}{bg=White,fg=nmsured}
\setbeamercolor{frametitle}{bg=White,fg=nmsured}
\setbeamercolor{section number projected}{bg=nmsured,fg=White}
\setbeamercolor{subsection number projected}{bg=nmsured,fg=White}
\setbeamertemplate{items}{\color{nmsured}{\faAngleDoubleRight}}
\setbeamertemplate{section in toc}[square]
\setbeamertemplate{subsection in toc}[square]
\setbeamertemplate{footline}[frame number]
\setbeamertemplate{caption}[numbered]
\setbeamerfont{footnote}{size=\tiny}
\setbeamercovered{invisible}
\usefonttheme{professionalfonts}
% set background
\setbeamertemplate{background}[grid][color=nmsured!15]
}

%
% title, author, date
%
\title{Dimuon Properties for Perfect Events}

\author{Dinupa}

\date{NMSU Update
\\ \today}


%
% some custom commands
%
\newcommand{\leftpic}[2]
{
\begin{textblock}{7.0}(0.5, 1.5)
\begin{figure}
    \centering
    \includegraphics[width=7.0cm]{../imgs/#1.png}
    \caption{#2}
\end{figure}
\end{textblock}
}

\newcommand{\rightpic}[2]
{
\begin{textblock}{7.0}(8.0, 1.5)
\begin{figure}
    \centering
    \includegraphics[width=7.0cm]{../imgs/#1.png}
    \caption{#2}
\end{figure}
\end{textblock}
}


\begin{document}

% make title page
\begin{frame}
    \maketitle
\end{frame}


%% slide 1
\begin{frame}[fragile]{Neural Network Architecture}

\begin{textblock}{14.0}(0.5, 2.0)
\begin{itemize}

    \item We use the same neural network architechture used in single tracks to extract the dimuon properties.
    
    \begin{itemize}
    
    \item We use classification layer and regression layer;
    
        \begin{scriptsize}
        \begin{verbatim}
        
* * * * * *     * * * * * * * * * * * * * * *     * * * * * * * * * * * * * * *     * * * * * *
* Input   *     * Classification Layer      *     * Regression layer          *     *         *
* Tensor  * --> * 2 Linear hidden layers    * --> * 3 Linear hidden layers    * --> * Target  *
*         *     * ReLu activation function  *     * ReLu activation function  *     * Tensor  *
* * * * * *     * CrossEntropyLoss          *     * MSELoss                   *     * * * * * *
                * Adam optimizer            *     * Adam optimizer            *
                * * * * * * * * * * * * * * *     * * * * * * * * * * * * * * *

Learning rate: 0.0001
L2 Regularization: 0.00001
        
        \end{verbatim}
        \end{scriptsize}
        
        \item We have to change the input \& output tensor features;
        
        \begin{scriptsize}
        \begin{verbatim}
For dimuons the input is;
        
[mu[/+, -/], station1[/x, y, z/], station3[/x, y, z/], station1[/px, py, pz], station3[/px, py, pz/]]
        
Target is;
        
[dimuon[/x, y, z, px, py, pz, m, pt, x1, x2, xf, costheta, phi/]]
        \end{verbatim}
        \end{scriptsize}
    
    \end{itemize}

\end{itemize}
\end{textblock}

\end{frame}

%% slide 2
\begin{frame}{Classification and Regression Loss}

\leftpic{cls-loss}{Train, validation classification loss and accuracy score.}

\rightpic{reg-loss}{Train , validation regression loss.}

\end{frame}


%% slide 3
\begin{frame}{Dimuon Properties}

\leftpic{px[GeV]}{Dimuon px.}

\rightpic{py[GeV]}{Dimuon py.}

\end{frame}


%% slide 4
\begin{frame}{Dimuon Properties}

\leftpic{pz[GeV]}{Dimuon pz.}

\rightpic{z[cm]}{Dimuon z.}

\end{frame}

%% slide 5
\begin{frame}{Dimuon Properties}

\leftpic{x[cm]}{Dimuon x.}

\rightpic{y[cm]}{Dimuon y.}

\end{frame}

%% slide 6
\begin{frame}{Dimuon Properties}

\leftpic{mass}{Dimuon mass.}

\rightpic{x1}{Dimuon x1.}

\end{frame}

%% slide 7
\begin{frame}{Dimuon Properties}

\leftpic{x2}{Dimuon x2.}

\rightpic{xF}{Dimuon xF.}

\end{frame}

%% slide 8
\begin{frame}{Dimuon Properties}

\leftpic{costheta}{Dimuon costheta.}

\rightpic{pT}{Dimuon pT.}

\end{frame}

%% slide 9
\begin{frame}{Dimuon Properties}

\leftpic{phi}{Dimuon phi.}

\begin{textblock}{7.0}(8.0, 1.5)
\begin{itemize}
    \item I'm still trying to figure out how to implement this method to real data. Suggestions ?
\end{itemize}
\end{textblock}

\end{frame}

\end{document}